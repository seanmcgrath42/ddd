Languages are messy. Irregular. Tempting to try to remove. Try to make everything homogenous.
I remember Intel verus 68000. Orthogal instruction set. The search for a perfect language.

	XOR AX,AX

Its short. Fast and memory efficient way of clearing a register. After a while, I got so used to it that I didn't even see it after a while. My brain went straight to "set AX to 0".

So it is with language shortcuts. Idioms. They are created precisely in the areas of most use.

Zipfs Law.

Rather than be horrified by them, useful to use them as guideposts to where the good stuff is in a corpus of code. Find the function that has 5 different meanings depening on context. There you will find the really important things. Signs of utility. If not useful, would not have developed shortcuts.

In music the term "seven" or "seventh" is a great example. The seven means the seventh note of the major scale, therefore the "7" in "C7" would surely mean the sevents note
of the major scale, right? No. It stands for the minor seventh - also know as flat seventh.

Why? Because the flat sevenths are more useful. They occur in increasing frequency as you move from folk to blues to Jazz.


